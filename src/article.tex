% BioMed_Central_Tex_Template_v1.05 %{{{1

\NeedsTeXFormat{LaTeX2e}[1995/12/01]
\documentclass[10pt]{bmc_article}    

% Load packages
\usepackage{cite} % Make references as [1-4], not [1,2,3,4]
\usepackage{url}  % Formatting web addresses  
\usepackage{ifthen}  % Conditional 
\usepackage{multicol}   %Columns
\usepackage[utf8]{inputenc} %unicode support
%\usepackage[applemac]{inputenc} %applemac support if unicode package fails
%\usepackage[latin1]{inputenc} %UNIX support if unicode package fails
\urlstyle{rm}
 
\def\includegraphic{}
\def\includegraphics{}

\setlength{\topmargin}{0.0cm}
\setlength{\textheight}{21.5cm}
\setlength{\oddsidemargin}{0cm} 
\setlength{\textwidth}{16.5cm}
\setlength{\columnsep}{0.6cm}

\newboolean{publ}

%Review style settings
\newenvironment{bmcformat}{\begin{raggedright}\baselineskip20pt\sloppy\setboolean{publ}{false}}{\end{raggedright}\baselineskip20pt\sloppy}

%Publication style settings
%\newenvironment{bmcformat}{\fussy\setboolean{publ}{true}}{\fussy}

% Scaffolder url
\urldef{\scaffolder}\url{http://next.gs}

% Begin ...
\begin{document}
\begin{bmcformat}

\title{Scaffolder - Software for Microbial Genome Scaffolding.} %{{{1

\author{
  Michael D Barton$^{1}$\correspondingauthor%
  \email{Michael D Barton\correspondingauthor  - mail@nucleotid.es}%
\and
  Hazel A Barton$^1$%
  \email{Hazel A Barton - bartonh@nku.edu}%
      }

\address{\iid(1) Department of Biological Sciences, Northern Kentucky %
University, Nunn Drive, Highland Heights, KY 41076 }%

\maketitle

\clearpage

\begin{abstract} %{{{1

  \paragraph*{Background:} Assembly of short read sequencing data can result in
  a fragmented series of contigs. Therefore a common step in a genome project
  is to join neighbouring sequence regions together and fill gaps in the
  assembly using additional sequence. This scaffolding step, however, is
  non-trivial and requires manually editing large blocks of nucleotide
  sequence. Joining sequence contigs together also hides the source of each
  region in the final genome sequence. Taken together, these considerations may
  make reproducing or editing an existing genome build difficult.

  \paragraph*{Methods:} The software outlined here ``Scaffolder'' is
  implemented in the Ruby programming language and can be installed via the
  RubyGems software management system. Genome scaffolds are defined using
  YAML - a markup language, which is both human and machine-readable. Command
  line binaries and extensive documentation are available.

  \paragraph*{Results:} This software allows a genome build to be defined in
  terms of the constituent contigs using a relatively simple to write and edit
  scaffold file syntax. This syntax further allows  unknown regions to be
  defined, and inserts to be added to the scaffold to fill gaps. Defining the
  genome construction in a file makes the scaffolding process reproducible and
  easier to edit and compare than fasta sequence.

  \paragraph*{Conclusions:} Scaffolder is easy to use genome scaffolding
  software. This tool promotes reproducibility and continuous development in a
  genome project. Scaffolder can be found at \scaffolder.

\end{abstract}

\ifthenelse{\boolean{publ}}{\begin{multicols}{2}}{}

\clearpage

\section*{Background} %{{{1
%{{{2

High throughput sequencing can produce hundreds of thousands to millions of
sequence reads from a genome. Each sequence represents the machine-read
nucleotides from a short region in the genome. At the time of writing, high
throughput sequencing is limited to producing reads less than 1000 nucleotides
in length. Therefore to resolve a sequence longer than this, such as a complete
genome, these numerous smaller fragments must be pieced together. The process
of joining reads into longer sequences is the `assembly' stage of a genome
project \cite{miller2010}. \pb

Assembly software takes the nucleotide reads produced by the sequencing
hardware and, in the ideal case, outputs a complete genome sequence composed of
these individual fragments. An analogy for this process is a jigsaw puzzle.
Each nucleotide read represents a single piece and the final genome sequence is
the completed puzzle. Sequences of repetitive nucleotide `repeat' regions
within the genome, or biased and incomplete sequencing data, may prevent the
genome being assembled into a continuous sequence. This may be due to not
enough or multiple overlaps between sequence reads and is analogous to missing
pieces in the jigsaw or pieces that fit to multiple other pieces. \pb

The recent advent of these current high throughput sequencing methods has lead
to a renewed interest into algorithms for solving the problem of genome
assembly \cite{pop2008,pop2009}; genome assembly software may however be unable
to produce a complete sequence from these reads. In many cases assembly
algorithms will generate several large contiguous regions of sequence
(`contigs') composed of the many available individual reads, which represent the
assembly of reads into the longest possible sequences given the data.
Nonetheless, these contigs represent a fragmented picture of the genome and
require additional work to join together into larger sequence lengths. \pb

The process of ameliorating a draft sequence into a finished continuous
sequence can be expensive in both time and laboratory effort, although the
genomic data present in a set of contigs may often be sufficient for many
research questions \cite{branscomb2002}. Nevertheless, a continuous
high-quality `finished' genome sequence does provide a greater depth of
information, such as complete resolution of repeat regions and precise
estimates of distances between genomic elements \cite{parkhill2002,fraser2002}.
This process of joining contigs together to form a continuous sequence is
called the `scaffolding' or `finishing' stage. \pb

\subsection*{Scaffolding} %{{{2

Scaffolding is the process of joining a series disconnected contigs into
complete continuous genome sequence. Due to genomic complexity and missing
data, scaffolding may not ultimately produce a final completed sequence, but
may still succeed in joining a greater subset of contigs together or resolving
gaps. An overview of required steps in the scaffolding process is outlined
below.

\subsubsection*{Contig Orientation} %{{{3

The sequencing process generates reads from either strand of the DNA helix, the
resulting contigs constructed from these reads may point in either direction of
the DNA. Orientating contigs to point in the same direction and represent the
same strand requires producing the reverse complement of sequences where
necessary. In the case of microbial genomes this orientation may be to the
5'-3' direction leading away from the origin of replication.

\subsubsection*{Contig Ordering} %{{{3

Contig ordering determines the placement of observed contigs to best represent
their order in the true genome sequence. The Correct placement of each contig
also highlights any extra-genomic DNA, such as plasmids which are scaffolded
separately from the genomic sequence. The contig order is commonly started at
the contig containing the origin of replication. All subsequent contigs are
then ordered in the 5' - 3' direction of DNA replication.

\subsubsection*{Contig Distancing} %{{{3

Given the correct order and orientation, determining the distance between
contigs results in an estimate of the complete genome size. The size of
inter-contig gaps represents the length of an unknown region in the genome.
Filling these regions with unknown nucleotide characters (`N') allows a draft
continuous sequence, which is useful for representing both the sequenced and
still to be resolved area in the genome sequence.

\subsubsection*{Gap Closing} %{{{3

During the scaffolding process closing and filling gaps between contigs
improves the completeness of the genome scaffold. Closing gaps may require
returning to the laboratory to perform additional sequencing or using
computational methods of estimating the unknown sequence. This additional
sequence is used to replace the gap between two contigs joining them into
a single sequence if they are not already scaffolded together. Once all contigs
have been joined and gaps in a scaffold closed, the genome may be considered
finished.

\subsection*{Computational Methods for Scaffolding} %{{{2

The process of resolving a genome sequence and building a scaffold uses
\emph{in vitro} methods, \emph{in silico} methods, or a combination of both. An
example of a computational method might use paired-read data, where the
sequencing reads are produced in pairs at a known distance apart in the genome.
The occurrence of a pair of reads in separate contigs can be used to estimate
the order and distance between the contigs. Alternatively, laboratory methods
may use PCR to amplify the unknown DNA in a gap region, then use traditional
Sanger sequencing to determine the gap sequence. Computational methods are more
preferable as they are less costly in laboratory time and materials compared to
manual gap resolution and uses the available data that exists following
sequencing \cite{nagarajan2010}. \pb

Synteny methods compare the assembled contigs to a more complete reference
genome sequence searching for areas of sequence similarity between the two.
Areas of corresponding sequence between the two sets are can be used to infer
contig placement and build the scaffold \cite{richter2007,zhao2008,assefa2009}.
Recombination between the two genomes can however reduce the effectiveness of
this analysis. Repeat regions may be responsible for many of the gaps in
building a genome sequence. The existence of tandemly repeated nucleotide
regions produces many similar reads in the sequence process. The assembly of
reads generated from repeat regions can result in the assemble collapsing into
a single repeat instance or ignored by conservative assembly algorithms. These
regions may be resolved by using reassembly algorithms specifically for repeat
resolution \cite{mulyukov2002,koren2010}. Similarly a related approach may use
unassembled sequence reads matching the regions around a scaffold gap to
construct a uniquely overlapping set of reads across the region
\cite{tsai2010}. \pb

Paired-read data provides an extra level of information on how contigs may be
scaffolded together. Heuristic algorithms may take advantage of this data to
search for optimal configuration of contigs in the scaffold using paired-read
distances \cite{dayarian2010,boetzer2011}. Additional data using a reference
genome may also be combined when estimating the best contig configuration
\cite{pop2004}. Finally when the scaffold cannot be completely resolved
\emph{in silico} software packages exist to suggest the likely primers
necessary for PCR amplifying the sequence in gap regions
\cite{gordon2001,nagarajan2010}. \pb

These methods provide a wide variety of possible approaches for scaffolding
a genome. The scaffold may still however require manual editing and addition of
sequence if computational methods are not completely successful. This involves
moving and joining larges blocks of nucleotide text together by hand which may
be error-prone and precludes any reproducibility. \pb

\subsection*{Overview} %{{{2

The software outlined here ``Scaffolder'' is aimed at creating a syntax and
software framework for producing and editing a genome scaffold. Scaffolder uses
a specific file format to define how contigs are joined, additional sequences
inserted, and for the specification of unknown regions. This syntax allows
a scaffold to be updated by simply editing the scaffold file and allows the
scaffolding process to be both reproducible and deterministic. \pb

\clearpage

\section*{Implementation} %{{{1

\subsection*{Code and Dependencies} %{{{2

Scaffolder is written in the Ruby programming language, version 1.8.7
\cite{ruby-lang}. The scaffolder package is split into two libraries. The first
called ``scaffolder'' contains the core Scaffolder application programming
interface (API). The second library ``scaffolder-tools'' contains the
Scaffolder command line interface (CLI). \pb

Unit tests were implemented to maintain individual elements of the source code
during development. Integration tests were written to test the Scaffolder
software interface as a whole. Unit tests were written using the Shoulda and
RSpec \cite{rspec} libraries.  Integration tests were written using the
Cucumber library \cite{rspec}. \pb

The Scaffolder source code is documented using the Yard library \cite{yard}.
Unix manual pages for the command line were generated using the Ronn library
\cite{ronn}. The manipulation of biological sequences in Scaffolder uses the
BioRuby library \cite{goto2010}. A full list of the software dependencies in
Scaffolder can be found in the Gemfile in the root of each source code
directory. \pb

\subsection*{Scaffold File Syntax} %{{{2

The choice of nucleotide sequences comprising the scaffold is specified using
the YAML syntax \cite{yaml}. YAML is a language format using whitespace and
indentation to produce a machine readable structure. As YAML is a standardised
data format, third-party developers have the option to generate a genome
scaffold using any programming language for which a YAML library exists. The
YAML website lists current parsers for languages including C/C++, Ruby, Python,
Java, Perl, C\#/.NET, PHP, and JavaScript. In addition to being widely
supported YAML formatted scaffold files can be validated for correct syntax
using third party tools such as Kwalify \cite{kwalify}. \pb

As described in the Background section, the initial sequencing data assembly
may result in an incomplete genome build. The use of computational software and
generation of additional sequence using PCR means genome scaffolding may be an
on-going process. Therefore the scaffold file should be simple to update
manually in addition to being computationally tractable. This requirement was
therefore the second reason YAML was selected, where YAML is human-readable and
simple to edit in a standard text editor. \pb

The scaffold file takes the form of a list of entries. Each entry corresponds
to a region of sequence, most likely a contig or previously scaffolded set of
contigs. The Scaffolder software processes this list in order, converting each
to the corresponding nucleotide sequence. These nucleotide sequences are then
concatenated to produce a draft super sequence. Each entry in the scaffold file
may have attributes that define whether a sub-sequence or the reverse
complement of the sequence should be used. The types of attributes available,
and an example scaffold file are outlined in the Results and Discussion section
below.  \pb

\clearpage

\section*{Results and Discussion} %{{{1

\subsection*{Scaffolder Simplifies Genome Finishing} %{{{2

The Scaffolder software outlined here allows computational joining of
nucleotide sequences together into a single contiguous super sequence.
Plain-text scaffold files written in the YAML syntax specify how these
sequences should be joined. This scaffold file is read by the scaffolder
software and used to generate the scaffold sequence. In addition to specifying
which contigs are required, the scaffold file allows the contigs to be edited
to produce smaller sub-sequences or reverse complemented if necessary. \pb

The process of genome finishing may involve producing additional
oligonucleotide sequence to fill unknown regions in the scaffold. The scaffold
file format allows these additional insert sequences to be added to bridge
these gaps. These inserts can also be treated in a similar way to larger contig
sequences: trimmed and/or reverse complemented to match the corresponding gap
region. \pb

Distances between contigs may be estimated from paired read data or from
mapping the contigs to a reference genome. These inter-contig gap regions are
useful to join regions of sequence together by an estimated size. The scaffold
file allows for the specification of these unknown regions to join contigs
together. The use of these regions in the scaffold highlights areas still to be
resolved. \pb

The scaffold file is used to specify how sequence regions are joined together
but the sequences themselves are not stored in the scaffold file. The sequences
are instead maintained as a separate fasta file. A fasta file of contigs and
scaffolded contigs is a common output for assemblers and as such should be
readily available. The nucleotide sequences in the fasta file are referenced in
the scaffold using the first space-delimited word from each fasta header.
Maintaining the nucleotide sequences in a separate file preserves the original
sequence and furthermore decouples the sequence from the markup describing how
it should be used to generate the genome sequence. \pb

\subsection*{Defining a Scaffold as a Text File} %{{{2

The scaffold file is written using the YAML syntax. YAML is a data format
designed to be both machine readable and easy to edit manually. An example
scaffold file is shown in Figure 1. This file illustrates the text attributes
used to describe a scaffold and the joining of sequences in the genome build.
The basic layout of the scaffold file is a list - a series of entries where
each entry corresponds to a region of sequence in the generated scaffold
super-sequence. The right hand side indicates how the entries are joined
together. This figure is illustrates the putative scaffold and is not to scale.
\pb

\subsubsection*{Simple sequence region} %{{{3

The first line of the scaffold file begins with three dashes. This is
a requirement to indicate the start of a YAML formatted document. The first
entry, highlighted in green, in the scaffold begins on the second line. The
first line of this entry begins with a dash character `-', dashes are used to
denote entries in a YAML list. The first line of each entry in the scaffold
therefore begins with a dash line. The next line is indented by two spaces
- whitespace is required to group similar attributes together. \pb

The ``sequence'' tag indicates this entry corresponds to a sequence in the
fasta file. The following line indicates the name of the fasta sequence using
the ``source'' tag. As described above the first space-delimited word of the
fasta header is used to identify which sequence is selected from the fasta
file. Together these three lines describe the first entry in the scaffold as
a simple sequence using the fasta sequence named `sequence1'. On the right hand
side of Figure 1 this produces the first region in the scaffold, also shown in
green. \pb

\subsubsection*{Unresolved sequence region} %{{{3

The second entry in the scaffold, highlighted in orange, is identified by the
``unresolved'' tag, indicating a region of unknown sequence but known length.
The second line specifies the size of this unknown region. This entry therefore
inserts a region of 20 `N' characters in the scaffold. \pb

\subsubsection*{Trimmed sequence region with multiple inserts} %{{{3

The last entry in the scaffold, highlighted in blue, adds the fasta entry
`sequence2' to the scaffold. This entry demonstrates how the nucleotide
sequence may be manipulated prior to addition to the scaffold. The `start' and
`stop' tags trim the sequence to these coordinates inclusively. The ``reverse''
tag instructs scaffolder to also reverse complement the sequence. On the right
hand side of the sequence of Figure 1 this completes the scaffold sequence. \pb

This final entry also includes the ``inserts'' tag to add additional regions of
sequence. These inserts are also added as a YAML list with each insert
beginning with a dash line. The first insert, shown in purple, illustrates
using similar attributes to that of a sequence entry. The reverse, start and
stop tags, are used to trim and reverse complement the insert. Similarly the
`source' tag identifies the corresponding fasta sequence. The ``open'' and
``close'' tags are specific to inserts and determine where the insert is added
in the enclosing sequence. The region of the sequence inside these coordinates
is inclusively replaced by the specified insert fasta sequence. This is
highlighted in the putative scaffold in Figure 1 by the black lines bisecting
the blue sequence. \pb

The next insert, shown in brown, is specified using only the open position tag.
This illustrates that only one of either `open' or `close' tags is required.
When only one of these tag is used the corresponding opposing coordinate
position is calculated from the length of the insert fasta sequence. This
allows inserts to bridge into, and partially fill gap regions without requiring
an end coordinate position. \pb

\subsubsection*{Scaffolder file attribute processing} %{{{3

The the scaffold file format allows multiple sequence-editing operations to be
applied to a sequence or insert. If applied in different combinations, these
operations will produce different output sequences. An example would be reverse
complementing the sequence then trimming the start of the sequence by five
nucleotides. Performing these actions in the reverse order would result in five
nucleotides being removed from the opposite end. This section therefore
outlines the Scaffolder protocol for processing sequence entries. \pb

\paragraph{Reverse sort inserts:} If the sequence entry contains a list of
inserts, these inserts are reverse sorted by the insert close coordinate
position. The insert with the largest close position is first in the list and
the insert with the smallest close position is last. This preserves the
original intended coordinates as each insert is added to the sequence. \pb

\paragraph{Update sequence with inserts:} Each insert sequence is processed
according to any specified attributes. The insert is then added to the
enclosing sequence in the region specified by the open and close coordinates.
\pb

If the insert added to the sequence is a different size to that of the
replaced region then the sequence start and stop co-ordinates are updated. For
example if a 7bp insert replaced a 5bp region in the sequence the stop
coordinate of the enclosing sequence would be incremented by the 2bp
difference. As inserts are added by reverse close position the remaining
inserts are unaffected by changes in the encoding sequence size. \pb

\paragraph{Trimming sequence:} If the sequence has start or stop tags
specified these are used as coordinates to trim the sequence to the specified
length inclusively. \pb 

\paragraph{Reverse complement sequence:} If the reverse tag is specified the
sequence is reverse complemented. This is final step in processing the
sequence and the sequence is then returned for inclusion in the scaffold. \pb

\subsection*{Scaffolder Usage} %{{{2

Scaffolder provides a standardised set of Ruby classes and methods (API) for
interacting with the scaffold. This allows Scaffolder to be integrated into
existing genomics workflows or used with Ruby build tools such as Rake. \pb

In addition Scaffolder provides a command line interface (CLI). This allows
the scaffold sequence to be generated and validated without any Ruby
experience. The Scaffolder CLI behaves as a standard Unix tool - returning
appropriate exit codes and providing manual pages. The use of both these
Scaffolder interfaces is outlined in the following sections. \pb

\subsubsection*{Installation} %{{{3

Scaffolder relies on both the Ruby programming language and the RubyGems
package management system. These are either already present or easily installed
on recent distributions of Mac OS X or Linux. The presence of both these tools
can be tested by typing the following commands on the terminal. \pb

  \begin{verbatim}
    ruby -v
    gem -v
  \end{verbatim}

If these commands return without error this indicates the necessary components
are available to install and run Scaffolder. The Scaffolder API and CLI can
then be installed using the following RubyGems command. \pb

  \begin{verbatim}
    gem install scaffolder scaffolder-tools
  \end{verbatim}

To ensure that the Scaffolder CLI is accessible, the RubyGems `bin' directory
should be in the command line path. Assuming that RubyGems are installed to
$\sim$/.gem this can be done by adding the following to the $\sim$/.bashrc
file. The corresponding login script should be updated for other shells such as
Zsh, etc.. \pb

  \begin{verbatim}
    export PATH=~/.gem/bin:$PATH
  \end{verbatim}

\subsubsection*{Command Line Interface} %{{{3

The Scaffolder CLI provides functions to validate the scaffold file and build
a draft super sequence. Assuming Scaffolder has been installed and the binary
is available in the command line path the following should run without error.
This command will list the available Scaffolder functions. \pb

  \begin{verbatim}
    scaffolder
  \end{verbatim}

Help on a specific command can be found by typing ``help'' followed by the
name of the command. This prints the corresponding Unix manual page to the
terminal. The following example will return the manual page for the sequence
command. \pb

  \begin{verbatim}
    scaffolder help sequence
  \end{verbatim} 

The ``sequence'' command builds the fasta sequence from the scaffold and fasta
file. Running this command requires the locations of both the scaffold file
and the fasta file to be passed as arguments. This will then return the
complete scaffold sequence in fasta format. \pb

  \begin{verbatim}
    scaffolder sequence scaffold.yml sequences.fna
  \end{verbatim}

Scaffolder provides a function to test the scaffold for overlapping insert
coordinates. Overlapping inserts will result in unexpected behaviour when
generating the scaffold sequence as the coordinates for the first insert may
change the intended coordinates for subsequent inserts. The ``validate''
command will check the scaffold for any such inserts and return them as a YAML
list. \pb

  \begin{verbatim}
    scaffolder validate scaffold.yml sequences.fna
  \end{verbatim}

\subsubsection*{Application Programming Interface} %{{{3

Scaffolder provides an API allowing Ruby programs to manipulate the scaffold.
This API is detailed in full on the Scaffolder website (\scaffolder). The
following code snippet outlines how the API may be used to create a script to
replicate the function of the `scaffolder sequence' command described in the
CLI section above. \pb

  \begin{verbatim}
    #!/usr/bin/env ruby

    require `rubygems'
    require `scaffolder'

    scaffold_file = ARGV.shift
    sequence_file = ARGV.shift

    scaffold = Scaffolder.new(
      YAML.load(File.read(scaffold_file)),
      sequence_file)

    sequence = scaffold.map{|entry| entry.sequence }.join

    puts sequence
  \end{verbatim}

This code snippet illustrates that the scaffold object is initialised with two
arguments: the YAML loaded scaffold and the location of the fasta sequence
file. The scaffold is an enumerable array and the penultimate line illustrates
the trivial code required to generate the super-sequence. Further details of
the classes and methods related to each scaffold entry can be found in the
documentation for the Scaffolder::Region class in the documentation on the
Scaffolder website. \pb

\subsection*{Discussion of Scaffolder Approach} %{{{2

The aims of Scaffolder are twofold. The first is providing software which is
easy to install and simplifies the task of genome finishing. The second aim is
to facilitate reproducibility when joining large nucleotide sequences together
to create a genome sequence. \pb

\subsubsection*{Simple to use} %{{{3

Scaffolder is designed to be as simple to use as possible. Assuming the Ruby
and RubyGems software are present, Scaffolder can be installed in a single
command line step. This should negate any possible barriers to entry which
require manual compilation of source code. \pb

Scaffolder uses a minimal and compact syntax to describe the genome scaffold.
This syntax is simple to construct and edit whilst being succinct and readable.
The goal of this syntax is to make genome scaffolds easy to construct in
a common text editor. Furthermore Scaffolder provides extensive documentation
both for the CLI and for the Ruby API. Therefore should any part of the
scaffolding process be unclear the documentation should make an solution easy
to find. \pb

\subsubsection*{Reproducible genome construction} %{{{3

Constructing a genome sequence using a text file makes generating a scaffold
super sequence both reproducible and deterministic. In comparison joining large
nucleotide sequences by hand cannot be reliably reproduced. The layout of the
scaffold file furthermore provides a human readable record of how the scaffold
was constructed and decouples the layout of contigs from their source sequence.
\pb

Configuring the final sequence in the scaffold file means the build is easier
to edit once already constructed. The YAML text format also allows comparison
of differences between scaffold builds using standard Unix tools such as diff.
This makes scaffold files easier to store in version control systems and allows
genome finishers methods similar to those in software development. \pb 

\clearpage

\section*{Conclusions} %{{{1

Scaffolder follows the Unix philosophy of ``Do one thing and do it well''. The
focus of Scaffolder is a simple to use tool for creating a genome scaffold. The
YAML data format for creating the scaffold file is standardised and easily
manipulated programmatically. This thereby means the scaffolding process
follows a second Unix tenet: ``If your data structures are good enough, the
algorithm to manipulate them should be trivial.''

\clearpage

\section*{Availability and Requirements} %{{{1

  \begin{description}
    \item[Project name:] Scaffolder
    \item[Project home page:] \scaffolder
    \item[Operating system:] Platform Independent. Tested on Mac OS X and
    Debian.
    \item[Programming language:] Ruby
    \item[Other requirements:] RubyGems
    \item[License:] MIT
    \item[Any restrictions to use by non-academics:] None
  \end{description}

\clearpage

\section*{Competing Interests} %{{{1

The authors declare no competing interests.

\section*{List of Abbreviations Used} %{{{1

  \begin{description}
    \item[API:] Application programming interface
    \item[CLI:] Command line interface
    \item[PCR:] Polymerase chain reaction
    \item[YAML:] YAML ain't markup language\cite{yaml}
  \end{description}

\section*{Authors contributions} %{{{1

MDB developed and maintains the Scaffolder tool. MDB and HAB wrote the
manuscript.

\section*{Acknowledgements} %{{{1
  \ifthenelse{\boolean{publ}}{\small}{}

This work was supported by the National Institute for Health: IDeA Network of
Biomedical Research Excellence (KY-INBRE) grant: NIH 2P20 RR016481-09 and the
NIH R15 AREA Program grant: NIH R15GM079775.

\clearpage

% Bibliography {{{1
{\ifthenelse{\boolean{publ}}{\footnotesize}{\small}
 \bibliographystyle{bmc_article}  % Style BST file
  \bibliography{article} }     % Bibliography file (usually '*.bib' ) 

\ifthenelse{\boolean{publ}}{\end{multicols}}{}

\clearpage

\section*{Figures} %{{{1

\subsection*{Figure 1 - Example of Scaffolder File and the Resulting Build}

An example scaffold file written using the YAML syntax \cite{yaml} (left) and
the resulting putative scaffold (right). The scaffold contains three entries
and two inserts. Each entry in the scaffold file text is delimited by a `-' on
a new line and highlighted using separate colours. The scaffold diagram on the
right is not to scale but illustrates how the scaffold sequences are joined.
\pb

\end{bmcformat}

\end{document}
